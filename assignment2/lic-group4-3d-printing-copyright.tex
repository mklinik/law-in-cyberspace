\documentclass[a4paper]{article}

% Font stuff
%\usepackage{fouriernc}
\usepackage{mathptmx}
\usepackage[T1]{fontenc}

% adds a toc to the pdf file, and makes refs clickable
\usepackage[bookmarks,colorlinks=true]{hyperref}

\begin{document}

\title{3D Printing, Copyright, Privacy and Data Protection}
\author{Markus Klinik
   \and Sergey Makarov
   \and Milton Hultgren
   \and Tom de Ruijter
   \and Ramon d. Graaff}
\maketitle

\begin{abstract}

This legal memorandum discusses the legal implications of 3D printing in view of
copyright, privacy and data protection.

\end{abstract}

\section{Background Information}

The act of copying is fundamental to the human nature.  From the days where apes
mimicked each other in how to crack nuts open with a stone, to children learning
from their parents how to use cook, Gutenberg's movable types, Xerox's
photocopier, and computers, whose most prominent feature it is to copy
information from the internet to memory, to the hard drive and to the internet
again, we've come a long way.

Over the past decade, a new technology arose which enables mankind to not only
copy information, but physical objects.  In the forseeable future, private
persons will be able to own a machine, roughly as big and expensive as a kitchen
espresso machine, that allows them to fabricate plastic objects like coat
hooks, replacement parts for washing machines, and even firearms at the press of
a button.

\end{document}

% vim: textwidth=80
