\documentclass[a4paper]{article}
\usepackage[english]{babel}

% Font stuff
%\usepackage{fouriernc}
\usepackage{mathptmx}
\usepackage[T1]{fontenc}

% for using unicode characters in the input document
\usepackage[utf8]{inputenc}

% adds a toc to the pdf file, and makes refs clickable
\usepackage[bookmarks,colorlinks=true]{hyperref}

\begin{document}

\title{3D Printing, Copyright, Privacy and Data Protection}
\author{Markus Klinik
   \and Sergey Makarov
   \and Milton Hultgren
   \and Tom de Ruijter
   \and Ramon de Graaff}
\maketitle

\begin{abstract}
This legal memorandum discusses the legal implications of 3D printing in view of
copyright, privacy and data protection.
We consider three different stages in 3D printing.
First there is the designer, creating a 3D blue print or 3D CAD file.
Secondly there is the distribution phase, where the blueprints are spread amongst user.
Lastly there is the usage phase where users print the actual object.
We regard the legal issues this new technology might bring forward from the perspective of copyright, and with a focus on the distribution phase we also look at 3D printing from the perspective of Privacy and Data Protection.
We close with a memorandum for courts, legislators and businesses in Europe and the Netherlands.

\end{abstract}

\section{Background Information}

The act of copying is fundamental to the human nature.  From the days where apes
mimicked each other in how to crack nuts open with a stone, to children
learning from their parents how to use a telephone, Gutenberg's movable types,
Xerox's photocopier, and computers, whose most prominent feature it is to copy
information from the internet to memory, to the hard drive and to the internet
again, we've come a long way.

Over the past decade, a new technology arose which enables mankind to copy not
only information, but physical objects.  In the forseeable future, private
persons will be able to own a machine, roughly as big and expensive as a
kitchen espresso machine, that allows them to fabricate plastic objects like
coat hooks, replacement parts for washing machines, and even firearms, at the
press of a button.

\subsection{The Act of Copying Something}

Copying information: which steps are there?  Scanning documents, transferring,
printing, reproducing etc.  How to copy text/information: Monks copying books
by hand, using the Xerox photocopier, scanning, uploading, downloading and
printing.

How to copy a physical object?  What's similar and what's different?
wood-cutting something by hand.  Scanning, uploading, downloading, printing.
Digital schematics.

If copying is so fundamental and important, why to restrict copying at all?
How does existing law restrict copying of documents/information?  Is it
specific to the thing that is being copied, or is it about the act of copying
"an sich"?  Does existing copyright law apply to physical objects?

\section{Advice For Judiciary}
When dealing with cases involving infringement of privacy and the processing of personal data it is of outmost importance that all steps of the process are performed with discretion and care. When addressed by a claimant of intellectual property rights seeking permission to infringe on the privacy of others the courts are advised to begin by ensuring that any data gathered so far does not fall within the classification of personal data as stated in Article 2.a 95/46/EC. If the claimant has already gathered personal data then they have infringed on the privacy of people without the permission of the judiciary and in doing so are violating Article 6 95/46/EC. As such all data gathered should be destroyed and any accusations based on said data should be seen as inadmissible. The claimant should be made aware of why this is and how they should gather their data in the future so that the new data is impersonal until a court order allows the claimant to do processing of personal data. It is up to the court to decide if the persons affected by such an infringement should be informed, to allow them to pursue a case of damages caused by the infringement. However, if the infringement has been light or informing the affected may hinder further proper investigations by the claimant then the court may choose to not inform the affected. 

3D printing opens up new challenges for intellectual property law but it still remains the task of the rights holder to present which of their intellectual property rights are being infringed upon by the sharing of blueprints or the printing with the use of said blueprints. There are many aspects for the claimant to consider and for the courts, in the later stages, to judge. Some blueprints and their resulting objects may fall under copyright protection, others fall under design protection. There are also patent rights, trademark rights and even trade secrets that may be part of these blueprints. It is up to the claimant to show the court if the blueprint alone is enough to prove that these rights have been infringed or if showing that the blueprint is present is a stepping stone to be granted a search warrant to find the printed version if it is only the printed version that infringes their rights.

Once the court agrees that the claimant has shown their intellectual property rights are being infringed upon and that said infringement is proportionate to the infringement of privacy of the persons the claimant seeks permission to perform, e.g. in terms of monetary loss, a court can grant the claimant permission for gathering additional evidence having this legitimate interest in mind. The courts should however not neglect to inform the claimant that this permission also comes with the obligation to notify the Supervisory Authority to inform them of the extent, intention and duration of their data processing and storing of said data as is their obligation in accordance with Article 18.1 95/46/EC.

After the claimant has carried out their processing, in accordance with the notification they have made to the Supervisory Authority and they have fulfilled their obligations they are free to use the data as evidence for the charges they wish to make. From there the case leaves the realm of privacy and data protection, as the infringement of privacy and data processing have been done in accordance with the law, and the case becomes an issue of intellectual property law and the claimant can use their evidence to try to prove that their rights have been infringed.


\section{Advice For Legistlature}

Do we need new laws?  When do we need new laws in general?   Does this apply to
3D printing? See Abel 2009 for an example.
http://www2.law.ed.ac.uk/ahrc/script-ed/vol6-1/abel.asp (Why did the german
court create a new law, instead of applying search and seizure to computers?
Why can courts create new law?)

\section{Advice For Industry}

\end{document}

% vim: textwidth=80
