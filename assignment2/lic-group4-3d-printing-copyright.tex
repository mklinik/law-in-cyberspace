\documentclass[a4paper]{article}
\usepackage[english]{babel}

% Font stuff
%\usepackage{fouriernc}
\usepackage{mathptmx}
\usepackage[T1]{fontenc}

% for using unicode characters in the input document
\usepackage[utf8]{inputenc}

% adds a toc to the pdf file, and makes refs clickable
\usepackage[bookmarks,colorlinks=true]{hyperref}

\begin{document}

\title{3D Printing, Copyright, Privacy and Data Protection}
\author{Markus Klinik
   \and Sergey Makarov
   \and Milton Hultgren
   \and Tom de Ruijter
   \and Ramon de Graaff}
\maketitle

\begin{abstract}
This legal memorandum discusses the legal implications of 3D printing in view of
copyright, privacy and data protection.
We consider three different stages in 3D printing.
First there is the designer, creating a 3D blue print or 3D CAD file.
Secondly there is the distribution phase, where the blueprints are spread amongst user.
Lastly there is the usage phase where users print the actual object.
We regard the legal issues this new technology might bring forward from the perspective of copyright, and with a focus on the distribution phase we also look at 3D printing from the perspective of Privacy and Data Protection.
We close with a memorandum for courts, legislators and businesses in Europe and the Netherlands.

\end{abstract}

\section{Background Information}

The act of copying is fundamental to the human nature.
From the days where apes mimicked each other in how to crack nuts open with a stone, to Gutenberg’s movable types, Xerox’s photocopier, and computers, amongst whose most prominent features it is to copy information from the internet to memory, to the hard drive and to the internet again, we’ve come a long way.

Over the past decades, a new technology arose which enables mankind to copy not only information, but physical objects.
In the foreseeable future, private persons will be able to own a machine, roughly as big and expensive as a coffee machine, that allows them to fabricate plastic objects like coat hooks, replacement parts for washing machines, and even firearms, at the press of a button.

\subsection{What is 3D Printing?}
A 3D Printer is a machine which, given a blueprint, produces a physical object according to the blueprint.
A blueprint is a computer file that describes the physical extension of an object in enough detail that the object can be reproduced.
There are three roles involved in the legal problem of 3D printing.
There is the \emph{user}, who uses a 3D printer to create a physical object from a given blueprint. There is the \emph{designer}, who creates blueprints.
If an existing physical object is involved, the party that may hold some IP right on the object will be called \emph{manufacturer}.
Take as an example Dave, who owns a car repair shop.
Dave has to repair the car for one of his customers, and he downloads the blueprint for the cap of the windshield wash from the internet.
He then uses his repair shop’s 3D printer to print the cap.
Dave is the user in this example.
The company who built the car may hold IP rights on the cap, so it is our manufacturer.
The person who created the blueprint for the cap, by either using an existing cap as reference, or by examining the tube it must be screwed onto, is the designer.

Designers can create blueprints in various ways, the most common are:
\begin{itemize}
  \item Using a 3D scanner to measure the spatial features of an existing object
  \item Using a CAD (Computer Aided Design) software to create blueprints by hand
\end{itemize}
For the latter option, the designer can use an existing object as reference or create a completely new blueprint without referring to an existing object.

3D printing on its own does not pose a legal problem.
It becomes a legal problem however, when the printed object is protected by 
some IP right, and printing happens against the will of the rights holder.

The act of 3D printing consists of three steps, each of which we will analyze 
independently.
\begin{enumerate}
  \item The designer creates a blueprint.
  \item The blueprint is distributed to the user.
  \item The user prints an object.
\end{enumerate}

If designer and user are the same person, there is no distribution step.
If the designer creates a new blueprint, say some piece of jewellery, without referring to some existing object, there is no legal problem.
We therefore consider the case where the object being printed is a copy of something that already exists and is being manufactured by usual industrial means.

In the following, the three steps are examined in detail, to analyze which IP laws could be violated in each case.
The jurisdiction under consideration is Dutch national law, and EU law when applicable.
We look at possible measures to prevent copying of objects, and how these measures interfere with privacy and data protection laws.

\subsection{Design}
For a blueprint to come into existence, there must be a person creating the blueprint in some way.
We examine which laws could possibly prohibit the creation of a blueprint from a given physical object.
The object, or certain aspects of it, may be protected by a variety of intellectual property rights.
Of particular interest are the dutch copyright (Auteursrecht) and design right (Modellenrecht).

Dutch copyright law applies to works of art and science such as texts, talks, software, photographs, films, recorded music, paintings, architecture and journalistic works\footnote{article 10, Nederlandse Auteurswet}.
%http://wetten.overheid.nl/BWBR0001886/geldigheidsdatum_22-10-2013#HoofdstukI_3_Artikel10
If the object in question is not a work of art, then copyright does not apply. Furthermore, there is to our understanding no clause which forbids detailed description in forms of blueprints.
Copyright law therefore does not preclude the creation of blueprints.

The Benelux design law excludes characteristics from being protected which are determined exclusively by technical function\footnote{Artikel 32, Beneluxverdrag inzake de intellectuele eigendom (merken en tekeningen of modellen), 's-Gravenhage, 25-02-2005}.
%http://wetten.overheid.nl/BWBV0001716/AuthentiekNL/VDRTKS559907/TITELIII/HOOFDSTUK1/Artikel32/geldigheidsdatum_22-10-2013
The most important use case of 3D printing is making spare parts to repair things.
The design of spare parts are usually determined by its function, so design law does not apply.

\subsection{Distribution} %Section by Ramon
As seen in the previous section, one may create a 3D-object by printing it from a blueprint. However, one can obtain a blueprint through various manners, where one of them is P2P file sharing. There have been a few lawsuits on P2P file sharing, but 3D-printing has never been explicitly mentioned, also due to it being a new technology. The question is how the EU legislation should be interpreted with respect to 3D printing. Can a blueprint be seen as intellectual property and how should the regulation with respect to P2P file sharing be interpreted?

Same type of intellectual property?
[case Escher??]

Several foundations such as BREIN and SABAM have filed cases against corporations such as Ziggo and NETLOG. In the opinion of the representatives for music, video and interactive departments, Ziggo (ISP) and NETLOG are used to share files with intellectual property. The prosecutors demanded actions against the illegal P2P file sharing which happened on their services. It is not unthinkable that corporations which use 3D-printing are going to file cases against such service providers.

The question is: “Does a blueprint of a 3D print fall under the same category, that is intellectual property?”. Although the design, a blueprint, of a 3D object is not the same as the object itself, since it is hard to send a 3D object over the digital net, we believe the blueprint does fall under the same category and can be treated the same.

Assuming a music file is treated the same as a blueprint, one might sue corporations which provide services such as P2P file sharing because of this new angle. Several lawsuits have been filed against these kind of service providers. For instance, an action demanded could be to install a monitoring system to filter for P2P file sharing of illegal files. In Case C-360/10 [SABAM vs NETLOG], an argument used was based on directive 2000/31/EC, the preamble, Recital 47: 

“Member States are prevented from imposing a monitoring obligation on service providers (…) of a general nature (…).”

This clearly states that filtering of a general nature is prohibited. In the same directive, Article 15(1), it is stated that: 

“Member States shall not impose a general obligation on providers, (...), to monitor the information which they transmit or store, nor a general obligation actively to seek facts or circumstances indicating illegal activity.”

This means that it is prohibited to seek for illegal activities in their traffic. Although, in Article 3 of Directive 2004/48/EC, which is closely related to directive 2001/29/EC, Article 8 (1), it’s stated that: 

“(1) Member States shall provide for the measures, procedures and remedies necessary to ensure the enforcement of the intellectual property rights covered by this Directive. Those measures, procedures and remedies shall be fair and equitable and shall not be unnecessarily complicated or costly, or entail unreasonable time-limits or unwarranted delays.
(2) Those measures, procedures and remedies shall also be effective, proportionate and dissuasive and shall be applied in such a manner as to avoid the creation of barriers to legitimate trade and to provide for safeguards against their abuse.”

This raises the question what effective, proportionate and dissuasive is, in this case. An injunction to require for service providers to install a filtering system would result in a “serious infringement of the freedom of the service provider to conduct its business since it would require that hosting service provider to install a complicated, costly, permanent computer system at its own expense”, which is not proportionate because there is no fair balance between the protection of the intellectual-property right enjoyed by copyright holders and on the other hand the freedom to conduct business enjoyed by service providers. Moreover, it would also be contrary to what is stated in Article 3(1) of Directive 2004/48/EC, which requires that measures to ensure the respect of intellectual-property rights should not be unnecessarily complicated or costly.

Moreover, installing a filtering system might undermine the freedom of information, since it might also filter lawful content, because of errors, which could lead to the blocking of lawful communications. Furthermore, by introducing a filtering system and therefore distinguishing between lawful and unlawful content might differ between Member States.

As stated in the judgment in Case C-275/06 Promusicae [2008] ECR I-271, the protection of the fundamental right to property, which includes the rights linked to intellectual property, must be balanced against the protection of other fundamental rights, which is in this case is Article 8 of the European Convention on Human Rights, which provides a right to respect for one's "private and family life, his home and his correspondence".

How is the intellectual property and copyright protection in Europe?

As a result, in both the cases SABAM versus NETLOG (Case C-360/10) and SABAM versus Scarlet (Case C-70/10), the result was:

“Directives 2000/31/EC, 2001/29/EC, 2004/48/EC, 95/46/EC and 2002/58/EC, read together and construed in the light of the requirements stemming from the protection of the applicable fundamental rights, must be interpreted as precluding an injunction made against an internet service provider which requires it to install a system for filtering information which is stored on its servers by its service users;
–        all electronic communications passing via its services, in particular those involving the use of peer-to-peer software;
–        which applies indiscriminately to all its customers;
–        as a preventative measure;
–        exclusively at its expense; and
–        for an unlimited period,
which is capable of identifying on that provider’s network the movement of electronic files containing a musical, cinematographic or audiovisual work in respect of which the applicant claims to hold intellectual-property rights, with a view to blocking the transfer of files the sharing of which infringes copyright.”

This means that the installation of a filtering system is prohibited. 

\subsection{Printing}

Private use

Article 2 95/46/EC
(a) 'personal data' shall mean any information relating to an identified or identifiable natural person ('data subject'); an identifiable person is one who can be identified, directly or indirectly, in particular by reference to an identification number or to one or more factors specific to his physical, physiological, mental, economic, cultural or social identity;
Article 6 95/46/EC
1. Member States shall provide that personal data must be:
(a) processed fairly and lawfully;
(b) collected for specified, explicit and legitimate purposes and not further processed in a way incompatible with those purposes. Further processing of data for historical, statistical or scientific purposes shall not be considered as incompatible provided that Member States provide appropriate safeguards;
(c) adequate, relevant and not excessive in relation to the purposes for which they are collected and/or further processed;
(d) accurate and, where necessary, kept up to date; every reasonable step must be taken to ensure that data which are inaccurate or incomplete, having regard to the purposes for which they were collected or for which they are further processed, are erased or rectified;
(e) kept in a form which permits identification of data subjects for no longer than is necessary for the purposes for which the data were collected or for which they are further processed. Member States shall lay down appropriate safeguards for personal data stored for longer periods for historical, statistical or scientific use.
2. It shall be for the controller to ensure that paragraph 1 is complied with.
Article 18 95/46/EC
Obligation to notify the supervisory authority
1. Member States shall provide that the controller or his representative, if any, must notify the supervisory authority referred to in Article 28 before carrying out any wholly or partly automatic processing operation or set of such operations intended to serve a single purpose or several related purposes.

\section{Advice For Judiciary}
When dealing with cases involving infringement of privacy and the processing of personal data it is of outmost importance that all steps of the process are performed with discretion and care.
When addressed by a claimant of intellectual property rights seeking permission to infringe on the privacy of others the courts are advised to begin by ensuring that any data gathered so far does not fall within the classification of personal data as stated in Article 2.a 95/46/EC.
If the claimant has already gathered personal data then they have infringed on the privacy of people without the permission of the judiciary and in doing so are violating Article 6 95/46/EC.
As such all data gathered should be destroyed and any accusations based on said data should be seen as inadmissible.
The claimant should be made aware of why this is and how they should gather their data in the future so that the new data is impersonal until a court order allows the claimant to do processing of personal data.
It is up to the court to decide if the persons affected by such an infringement should be informed, to allow them to pursue a case of damages caused by the infringement.
However, if the infringement has been light or informing the affected may hinder further proper investigations by the claimant then the court may choose to not inform the affected.

3D printing opens up new challenges for intellectual property law but it still remains the task of the rights holder to present which of their intellectual property rights are being infringed upon by the sharing of blueprints or the printing with the use of said blueprints.
There are many aspects for the claimant to consider and for the courts, in the later stages, to judge.
Some blueprints and their resulting objects may fall under copyright protection, others fall under design protection.
There are also patent rights, trademark rights and even trade secrets that may be part of these blueprints.
It is up to the claimant to show the court if the blueprint alone is enough to prove that these rights have been infringed or if showing that the blueprint is present is a stepping stone to be granted a search warrant to find the printed version if it is only the printed version that infringes their rights.

Once the court agrees that the claimant has shown their intellectual property rights are being infringed upon and that said infringement is proportionate to the infringement of privacy of the persons the claimant seeks permission to perform, e.g. in terms of monetary loss, a court can grant the claimant permission for gathering additional evidence having this legitimate interest in mind.
The courts should however not neglect to inform the claimant that this permission also comes with the obligation to notify the Supervisory Authority to inform them of the extent, intention and duration of their data processing and storing of said data as is their obligation in accordance with Article 18.1 95/46/EC.

After the claimant has carried out their processing, in accordance with the notification they have made to the Supervisory Authority and they have fulfilled their obligations they are free to use the data as evidence for the charges they wish to make.
From there the case leaves the realm of privacy and data protection, as the infringement of privacy and data processing have been done in accordance with the law, and the case becomes an issue of intellectual property law and the claimant can use their evidence to try to prove that their rights have been infringed.

\section{Advice For Legistlature}

We have defined 3D printing to consist of three different steps: creation, 
distribution and usage (printing) of 3D CAD files.
In this section we will look at these steps from the perspective of intellectual
 property law, privacy and personal data protection, with a focus on the 
 distribution step.

%// TODO: explain structure in rest of chapter.

\subsection{Analysis}

\paragraph{Copyright Law}
Firstly we will analyse how Dutch and European copyright law can apply to 3D CAD
 files.
Can a file itself be copyrighted, or only the object it represents or contains? 
This is comparable to intellectual property of artistic work and the technical 
measures to prevent their copying, such as DRM.
Finally one might wonder if a printed object might be copyrighted.

\paragraph{Copyright on files}
As sketched in the introductory section, whether or not intellectual property 
law applies on an object or file depends on the contents of that file.
A file in its purest form can be seen as a representation of a virtual object or
 idea.
A 3D CAD file is a specific virtual representation of an object, possibly 
encapsulating esthetical or patented functional ideas.

Without disproportionately difficult to implement technical measures, it can be 
very difficult to enforce copyright on files without regarding the content.
Such technical measures would certify said program to be related to some 
protected content without directly regarding the contents of a file.
An example of this is DRM, the failed attempt of the music industry to protect 
artists’ interests.
Failed, in the sense that imposed technical measures did not prevent replication
 and unlawful sharing.
A more interesting purpose for such technical measures is not to prevent 
copying, but to inform users that the file’s contents are copyright protected.

In no circumstances can a CAD file be considered software as it does not 
specifically instruct a machine how to operate and therefore Dutch and European 
software law does not apply.
%[http://wetten.overheid.nl/BWBR0001886/geldigheidsdatum_22-10-2013#HoofdstukI_3_Artikel10 12°] .
This is similar to the argument that a 2D photograph cannot be considered 
software.
Even if 3D CAD files would be protected through software law, the ECJ ruled 
that 
\begin{quote}``The purchaser of a licence for a program is entitled, as a rule, to 
observe, study or test its functioning so as to determine the ideas and 
principles which underlie that program.'' [Case C-406/10]\end{quote}
The question then becomes whether these ideas and principles are protected.

\paragraph{Copyright on ideas}
This question extends to the domain of 3D printing files.
When considering intellectual property on 3D CAD files, legislation should only 
encompass ideas and underlying design principles.
Apart from serving a completely artistic purpose, such as a musical work, 3D 
printing often serves a more utilitarian goal as acquiring tools and simple 
spare components no longer requires specialist 3rd parties.
Currently, Dutch and European IP law excludes certain utilitarian objects from 
copyright.
%[http://wetten.overheid.nl/BWBV0001716/AuthentiekNL/VDRTKS559907/TITELIII/HOOFDSTUK1/Artikel32/geldigheidsdatum_22-10-2013]
If an innovative step is included in the design, Dutch patent law might be 
applicable instead.
%[http://wetten.overheid.nl/BWBR0007118/Hoofdstuk1/Artikel2/geldigheidsdatum_20-10-2013].

\paragraph{Copyright on printed objects}
% TODO: I couldn’t find any legislation on what happens if an object that is printed is copyrighted.

\paragraph{Data protection and takedown notices}
As 3D printing is a new technology, not much case law exists on copyright 
takedown notices for unlawfully shared 3D CAD files and printed objects.
One of the first of such cases is based on a Digital Millenium Copyright Act 
(DMCA) takedown notice in the U.S.A.
%[Brian Rideout, Printing the Impossible Triangle: the Copyright Implications of Three- Dimensional Printing, 5 Journal of Business Entrepreneurship & Law, 2011-2012, p. 161- 177.]
Thingiverse, a network for sharing 3D CAD files, is one of the parties of 
concern in the case.
Supposably, one of Thingiverse's users copied original content of another user.
In the end Thingiverse removed the appointed files, even though the claimant
already withdrew the notice.

We have shown that at this moment private use of 3D CAD files is not unlawful.
The question of interest is how intellectual property law applies on 3D printing
 in such networks and how hosting networks and their users can continue to 
 safely share their ideas with the public as printing technology becomes more 
widespread.

\subsection{Expanding legislation}
Based on these grounds, several options exist for expanding legislation.

Copyright law
- Impose bounds on 3D printer market and usage of printers (*)
- Impose bounds on scanning of real-life objects

- Impose bounds on copyright of 3D CAD files 
- Impose bounds on distribution of 3D CAD files
	- Make sharing networks liable for actions and content posted by users (Kazaa case - See Milton’s summary) (*)
	- Make it possible for 3rd party authorities to monitor and control content posted by users and possible prosecute said users
- Loosen bounds on distribution of 3D CAD files

- Impose stricter bounds on copyright of 3D printed objects (Similar to copying of books)

- Should legislation make an exception for home copies?

\section{Advice For Industry}
For industry, 3D printing has both up- and downsides.
On the one hand, companies can use this technology for prototyping, research and manufacturing on demand, but on the other hand 3D printers can and will be used by individuals to print objects protected by IP law.
The main issue is illegal distribution of blueprints created by the company itself, which are protected by some form of IP law.

Unwanted distribution of digital blueprints is similar to the distribution of other content like music, films or software, and it could be useful for companies who worry about their blueprints to look at the developments in these areas.

One measure to migitate the problem would be to prevent leaking of blueprints in the first place.
There should be defined processes for handling of blueprints, which don't leave the possibility for people outside the company to get their hands on them.
If the company has a good security system, the chance of blueprints leaking to the internet is rather small.

If these methods don't work and for some reason blueprints eventually leak to the internet, there are ways for companies to obstruct their distribution. That, however, needs to be done carefully, because some of the techniques may infringe the law.
One method is to track IPs of users who illegally download or distribute blueprints, and file private lawsuits against them.
Companies should be aware that this may infringe the privacy of users, thus being unlawful.
An IP address in isolation is not personal data because it identifies a computer and not an individual person.
With the help of an ISP though, an IP address becomes personal data when combined with other information such as names and address of users.
To address if copyright overrides the right to privacy we refer to ECHR article 8 which states that the right can be overridden when it is ``necessary [...] in the interests of [...] the economic well-being of the country".
The economic well-being of a country depends to a large extent on the economic well-being of its private companies, so this exception may apply here.

To enforce such claims, right holders first need a court ruling telling the ISP to disclose information which associates IP addresses and individuals.
Beyond this legal ruling, right holders also need to notify the supervisory authority in their state about the details of this processing of personal data and provide names and contact information as well as a list of measures taken to ensure the security of the processing.
This is to be reviewed to see if the rights and freedoms of the data subject are being violated unduly with regard to data protection law.
In light of these safeguards, if right holders follow proper procedures, the courts should have no reason to dismiss their claims.

Another way is to make agreements with websites, where blueprints are illegally distributed, to delete those and restrict users from uploading them again.
If that doesn't help and websites don't cooperate, then companies could ask internet service providers to block access to these websites.
There is no common opinion throughout EU yet though, whether ISPs should block access to such websites on the demand of copyright holders.
Courts of some countries, like UK or Finland, oblige ISPs to do it, while others, like Germany, don't.
It might also be possible to not block the whole website, but filter the data and block only that content which infringes copyright.

An alternative way for industry could be to not fight against thousands of users with lawsuits and websites by trying to block them, but to provide distribution services themselves.
Let's look at the past decade with respect to the difficult times the music industry had because of illegal file sharing.
The measures they took to regain control over their old business model alienated customers and gave fuel to the arguments of file sharing advocates.
The music industry made investments to implement gapless digital rights management from distribution to end user devices, but that didn't impact file sharing networks, and created inconvenience for paying customers.
It took some years, and Apple eating a big portion of their lunch, for the music industry to embrace the internet as a new form of distribution network for themselves.
A lesson that the film industry has yet to learn.

By providing high quality blueprints, say as charged downloads, as a legal alternative, which has to be at least as convenient as state of the art peer to peer file sharing, companies could try to claim this new market before others do.
Supplying parts this way, companies can save the costs for manufacturing, storage and shipment of physical goods.

It could also be conceivable, considering specifics of 3D-printing, to only allow users to print the item once or some other limited amount of times.

\end{document}

% vim: textwidth=80
