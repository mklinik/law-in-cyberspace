\section{Advice For Legistlature}

We have defined 3D printing to consist of three different steps: creation, distribution and usage (printing) of 3D CAD files. In this section we will look at these steps from the perspective of copyright and intellectual property law, privacy and personal data protection, with a focus on the distribution step.

// TODO: explain structure in rest of chapter.

\subsection{Analysis}

\paragraph{Copyright Law}
Firstly we will analyse how Dutch and European intellectual property rights can apply to 3D CAD files. 

Can a file itself be copyrighted, or only the object it represents or contains? This is comparable to intellectual property of artistic work and the technical measures to prevent their copying, such as DRM.
Finally one might wonder if a printed object might be copyrighted.

\paragraph{Copyright on files}
As sketched in the introductory section, whether or not intellectual property law applies on an object or file depends on the nature of that file. A file in its purest form can be seen as a representation of a virtual object or idea. A 3D CAD file is a specific virtual representation of an object, possibly encapsulating esthetical or patented functional ideas.

Without disproportionately difficult technical measures, it can be very difficult to enforce copyright on files without regarding the content. Such technical measures would certify said program to be related to some protected content without directly regarding the contents of a file. An example of this is DRM, the failed attempt of the music industry to protect artists’ interests. Failed, in the sense that imposed technical measures did not prevent replication and unlawful sharing. A more interesting purpose for such technical measures is not to prevent copying, but to inform users that the file’s contents are copyright protected.

In no circumstances can a CAD file be considered software as it does not specifically instruct a machine how to operate and therefore Dutch and European software law does not apply. %[http://wetten.overheid.nl/BWBR0001886/geldigheidsdatum_22-10-2013#HoofdstukI_3_Artikel10 12°] .
This is similar to the argument that a 2D photograph cannot be considered software. Even if 3D CAD files would be protected through software law, the ECJ ruled that “The purchaser of a licence for a program is entitled, as a rule, to observe, study or test its functioning so as to determine the ideas and principles which underlie that program.” [Case C-406/10]. The question then becomes whether these ideas and principles are protected.

\paragraph{Copyright on ideas}
This question extends to the domain of 3D printing files. When considering intellectual property on 3D CAD files, legislation should only encompass ideas and underlying design principles.

Apart from serving a completely artistic purpose, such as a musical work, 3D printing often serves a more utilitarian goal as acquiring tools and simple spare components no longer requires specialist 3rd parties. Currently, Dutch and European IP law excludes certain utilitarian objects from copyright.  %[http://wetten.overheid.nl/BWBV0001716/AuthentiekNL/VDRTKS559907/TITELIII/HOOFDSTUK1/Artikel32/geldigheidsdatum_22-10-2013]
If an innovative step is included in the design, Dutch patent law might be applicable instead. % [http://wetten.overheid.nl/BWBR0007118/Hoofdstuk1/Artikel2/geldigheidsdatum_20-10-2013].

\paragraph{Copyright on printed objects}
// TODO: I couldn’t find any legislation on what happens if an object that is printed is copyrighted.

\paragraph{Data protection and takedown notices}
As 3D printing is a new technology, not much case law exists on copyright takedown notices for unlawfully shared 3D CAD files and printed objects. One of the first of such cases is based on a Digital Millenium Copyright Act (DMCA) takedown notice in the U.S.A. %[Brian Rideout, Printing the Impossible Triangle: the Copyright Implications of Three- Dimensional Printing, 5 Journal of Business Entrepreneurship & Law, 2011-2012, p. 161- 177.]
Thingiverse, a network for sharing 3D CAD files, is one of the parties of concern in the case. Supposably, one user copied original content of another user on this network. In the end Thingiverse removed the appointed files, even though the claimant already withdrew the notice. 

We have shown that at this moment private use of 3D CAD files is not unlawful.
The question of interest is how intellectual property law applies on 3D printing in such networks and how hosting networks and their users can continue to safely share their ideas with the public as printing technology becomes more widespread.

\subsection{Expanding legislation}
Based on these grounds, several options exist for expanding legislation.

Copyright law
- Impose bounds on 3D printer market and usage of printers (*)
- Impose bounds on scanning of real-life objects

- Impose bounds on copyright of 3D CAD files 
- Impose bounds on distribution of 3D CAD files
	- Make sharing networks liable for actions and content posted by users (Kazaa case - See Milton’s summary) (*)
	- Make it possible for 3rd party authorities to monitor and control content posted by users and possible prosecute said users
- Loosen bounds on distribution of 3D CAD files

- Impose stricter bounds on copyright of 3D printed objects (Similar to copying of books)

- Should legislation make an exception for home copies?
