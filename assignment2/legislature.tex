\section{Advice For Legistlature}

We have defined 3D printing to consist of three different steps: creation, 
distribution and usage (printing) of 3D CAD files.
In this section we will look at these steps from the perspective of intellectual
 property law, privacy and personal data protection, with a focus on the 
 distribution step.

%// TODO: explain structure in rest of chapter.

\subsection{Analysis}

\paragraph{Copyright Law}
Firstly we will analyse how Dutch and European copyright law can apply to 3D CAD
 files.
Can a file itself be copyrighted, or only the object it represents or contains? 
This is comparable to intellectual property of artistic work and the technical 
measures to prevent their copying, such as DRM.
Finally one might wonder if a printed object might be copyrighted.

\paragraph{Copyright on files}
As sketched in the introductory section, whether or not intellectual property 
law applies on an object or file depends on the contents of that file.
A file in its purest form can be seen as a representation of a virtual object or
 idea.
A 3D CAD file is a specific virtual representation of an object, possibly 
encapsulating aesthetic or patented functional ideas.

Without disproportionately difficult to implement technical measures, it can be 
very difficult to enforce copyright on files without regarding the content.
Such technical measures would certify said program to be related to some 
protected content without directly regarding the contents of a file.
An example of this is DRM, the failed attempt of the music industry to protect 
artists's interests.
Failed, in the sense that imposed technical measures did not prevent replication
 and unlawful sharing.
A more interesting purpose for such technical measures is not to prevent 
copying, but to inform users that the file’s contents are copyright protected.

In no circumstances can a CAD file be considered software as it does not 
specifically instruct a machine how to operate and therefore Dutch and European 
software law does not apply \footnote{Artikel 10, Lid 12, Nederlandse Auteurswet}.
This is similar to the argument that a 2D photograph cannot be considered 
software.
Even if 3D CAD files would be protected through software law, the ECJ ruled 
that 
\begin{quote}``The purchaser of a licence for a program is entitled, as a rule, to 
observe, study or test its functioning so as to determine the ideas and 
principles which underlie that program.'' [Case C-406/10]\end{quote}
The question then becomes whether these ideas and principles are protected.

\paragraph{Copyright on ideas}
This question extends to the domain of 3D printing files.
When considering intellectual property on 3D CAD files, legislation should only 
encompass ideas and underlying design principles.
Apart from serving a completely artistic purpose, such as a musical work, 3D 
printing often serves a more utilitarian goal as acquiring tools and simple 
spare components no longer requires specialist 3rd parties.
Currently, Dutch and European IP law excludes certain utilitarian objects from 
copyright\footnote{Artikel 32, Beneluxverdrag inzake de intellectuele eigendom (merken en tekeningen of modellen), 's-Gravenhage, 25-02-2005}.
If an innovative step is included in the design, Dutch patent law might be 
applicable instead\footnote{Artikel 2, Rijksoctrooiwet 1995}.
No private purpose exception exists in the Netherlands for patented objects.

\paragraph{Copyright on printed objects}
A last option is that blueprints are lawful, but the resulting printed object is not.
We just established that blueprints are descriptions of virtual objects.
Perfume bottles are often copyrighted for their unique designs.
There is no law preventing one to create a blueprint based on publicly available material of such a bottle as it is only an accurate description.
However to our knowledge, no specific laws exist yet for sharing blueprints of copyrighted objects.

\paragraph{Data protection and takedown notices}
As 3D printing is a new technology, not much case law exists on copyright 
takedown notices for unlawfully shared 3D CAD files and printed objects.
One of the first of such cases is based on a Digital Millenium Copyright Act 
(DMCA) takedown notice in the U.S.A.\footnote{Brian Rideout, ``Printing the Impossible Triangle: the Copyright Implications of Three-Dimensional Printing'', Journal of Business, Entrepreneurship \& Law 5 (2011): 161-177.}
Thingiverse, a network for sharing 3D CAD files, is one of the parties of 
concern in the case.
Supposably, one of Thingiverse's users copied original content of another user.
In the end Thingiverse removed the appointed files, even though the claimant
already withdrew the notice.

We have shown that at this moment private use of 3D CAD files is not unlawful.
The question of interest is how intellectual property law applies on 3D printing
 in such networks and how hosting networks and their users can continue to 
 safely share their ideas with the public as printing technology becomes more 
widespread.

\subsection{Expanding legislation}
Based on the previous, several options exist for expanding legislation.
Within public law, there is the option of posing bounds on possession of 3D printing technology.
Printing technology could be used for printing lethal weapons\footnote{``Bill Maher Sounds Off On 3D-Printed Guns'', read on October 10, 2013, http://www.huffingtonpost.com/tag/3d-printer-gun}.
However, this contradicts with the European Directive on eCommerce, as sharing of blueprints cannot be limited to the borders of a single country\footnote{Directive 2000/31/EC of the European Parliament and of the Council of 8 June 2000 on certain legal aspects of information society services, in particular electronic commerce, in the Internal Market (`Directive on electronic commerce')} and this heavily limits open market development.

Current law does not fully cover the current nature of 3D printing, as the contents of a 3D CAD file diverge heavily.
When the nature of a blueprint is artistic, this case is similar to the P2P file sharing cases that currently run throughout Europe.
Also when the modelled object is artistic, similar law applies.
However, often a blueprint encapsulates a utilitarian object for which current Dutch patent laws do not suffice.
Also there exists no home copy legislation for such objects.

We strongly recommend to reduce the amount of friction between innovation and commercial benefit.
Contrary to music file sharing, we believe sharing of blueprints can create a climate with rapid technological improvements and increased quality of life for individuals.

Current blueprint sharing networks do not utilise a P2P strategy, but rather host all files in full. 
An option is to make these networks liable for actions and content posted by users, similar to the IP landmark Napster case in 2001\footnote{A\&M RECORDS, Inc. v. NAPSTER, INC., 239 F.3d 1004 (9th Cir. 2001)}, where Napster, the company behind the file sharing program Napster, was the first party to be held fully liable for mass-scale digital copyright infringement.

Imposing 3rd party monitoring, as is the case in the U.S.A. with the voluntary six-strikes agreement between ISPs\footnote{``Joint Strategic Plan on Intellectual Property Enforcement'', U.S. Intellectual Property Enforcement Coordinator, June 2013}, compromises user privacy and their personal data.
It is also not as effective as targeting the sharing platform itself.

We would finally like to recommend legislators to promote openness of designing and 3D printing of objects, with open source and creative commons licenses, as this avoids legislation altogether.
