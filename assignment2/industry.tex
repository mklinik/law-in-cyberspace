\section{Advice For Industry}
For industry, 3D printing has both up- and downsides.
On the one hand, companies can use this technology for prototyping, research and manufacturing on demand, but on the other hand 3D printers can and will be used by individuals to print objects protected by IP law.
The main issue is illegal distribution of blueprints created by the company itself, which are protected by some form of IP law.

Unwanted distribution of digital blueprints is similar to the distribution of other content like music, films or software, and it could be useful for companies who worry about their blueprints to look at the developments in these areas.

One measure to migitate the problem would be to prevent leaking of blueprints in the first place.
There should be defined processes for handling of blueprints, which don't leave the possibility for people outside the company to get their hands on them.
If the company has a good security system, the chance of blueprints leaking to the internet is rather small.

If these methods don't work and for some reason blueprints eventually leak to the internet, there are ways for companies to obstruct their distribution. That, however, needs to be done carefully, because some of the techniques may infringe the law.
One method is to track IPs of users who illegally download or distribute blueprints, and file private lawsuits against them.
Companies should be aware that this may infringe the privacy of users, thus being unlawful.
An IP address in isolation is not personal data because it identifies a computer and not an individual person.
With the help of an ISP though, an IP address becomes personal data when combined with other information such as names and address of users.
To address if copyright overrides the right to privacy we refer to ECHR article 8 which states that the right can be overridden when it is ``necessary [...] in the interests of [...] the economic well-being of the country".
The economic well-being of a country depends to a large extent on the economic well-being of its private companies, so this exception may apply here.

To enforce such claims, right holders first need a court ruling telling the ISP to disclose information which associates IP addresses and individuals.
Beyond this legal ruling, right holders also need to notify the supervisory authority in their state about the details of this processing of personal data and provide names and contact information as well as a list of measures taken to ensure the security of the processing.
This is to be reviewed to see if the rights and freedoms of the data subject are being violated unduly with regard to data protection law.
In light of these safeguards, if right holders follow proper procedures, the courts should have no reason to dismiss their claims.

Another way is to make agreements with websites, where blueprints are illegally distributed, to delete those and restrict users from uploading them again.
If that doesn't help and websites don't cooperate, then companies could ask internet service providers to block access to these websites.
There is no common opinion throughout EU yet though, whether ISPs should block access to such websites on the demand of copyright holders.
Courts of some countries, like UK or Finland, oblige ISPs to do it, while others, like Germany, don't.
It might also be possible to not block the whole website, but filter the data and block only that content which infringes copyright.

An alternative way for industry could be to not fight against thousands of users with lawsuits and websites by trying to block them, but to provide distribution services themselves.
Let's look at the past decade with respect to the difficult times the music industry had because of illegal file sharing.
The measures they took to regain control over their old business model alienated customers and gave fuel to the arguments of file sharing advocates.
The music industry made investments to implement gapless digital rights management from distribution to end user devices, but that didn't impact file sharing networks, and created inconvenience for paying customers.
It took some years, and Apple eating a big portion of their lunch, for the music industry to embrace the internet as a new form of distribution network for themselves.
A lesson that the film industry has yet to learn.

By providing high quality blueprints, say as charged downloads, as a legal alternative, which has to be at least as convenient as state of the art peer to peer file sharing, companies could try to claim this new market before others do.
Supplying parts this way, companies can save the costs for manufacturing, storage and shipment of physical goods.

It could also be conceivable, considering specifics of 3D-printing, to only allow users to print the item once or some other limited amount of times.
