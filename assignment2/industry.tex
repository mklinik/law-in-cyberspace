\section{Advice For Industry}
For industry, 3D printing has both up and downsides.
On the one hand, companies can successfully use this technology for prototyping, research purposes and distributed manufacturing, but on the other hand spreading of 3D printers can and will cause data protection and privacy problems.
The main issue is illegal distribution of blueprints that are protected under copyright law.

The problem of blueprints distribution is quite similar to the problem of distribution of any other content - music, films, software, etc., so it is useful for companies, who worry for their blueprints, to look at their colleagues experience in these areas.
First of all, it is important to prevent stealing and further illegal distribution inside the company.
There should be safe processes of handling the blueprints, which don’t give any possibility to steal it for people outside the company or without right to use these blueprints.
If the company has a good security system, then chances of blueprints to be leaked to the Internet are very small and the company wouldn’t have to spend so much resources, struggling against websites and users.

If these methods didn’t work or for any other reason blueprints eventually got into the Internet, there are some ways for companies to obstruct its distribution, but it needs to be done carefully, because some of the remedies may infringe the law.
One method is to track IPs of the users, who illegally download or distribute these blueprints.
But companies should be aware that this way may infringe users privacy and be unlawful.
An IP address in isolation is not personal data because it is focused on a computer and not an individual.
Though, in the hands of an ISP an IP address becomes personal data when combined with other information - user’s name and address.
To address if copyright override the right of privacy we can refer to ECHR article 8 that states that the right can be overridden when it’s “necessary in a democratic society in the interests of national security, public safety or the economic well-being of the country, for the prevention of disorder or crime, for the protection of health or morals, or for the protection of the rights and freedoms of others”.
[http://www.hri.org/docs/ECHR50.html] In our context the economic well-being of the country is a key as trade in copyrighted material is a big part of the economy.
The next issue is how to enforce these aims.
Any rights holder first needs a court ruling telling the ISP to disclose the information.
Beyond this legal ruling the right holder will also need to notify the supervisory authority in their state about the details of this new processing and provide names and contact information as well as a list of measures taken to ensure the security of the processing.
This is to be reviewed to see if the rights and freedoms of the data subject are being violated unduly with regard to for example data quality(specific purpose, time stored etc).
In light of these safeguards if right holders follow proper procedures the courts should have no reason to dismiss their claims.

Another way is to demand websites, where the blueprints are illegally distributed, to delete it and restrict users from re-uploading it.
If it doesn’t help and websites don’t obey, then companies should ask the Internet Service Providers to block the access to these websites.
Though, there is no common opinion throughout EU yet - should ISPs block access to the websites by copyright holders demand or not.
Courts of some countries, like UK or Finland, oblige ISPs to do it, when others, like Germany, don’t.
It also may be a good way to block not the whole website, but filter the data and block only that content, which infringe copyright.

But the best way for industry is not to struggle against thousands of users by suing them and websites by trying to block it, but to provide a good distribution service for blueprints with reasonable prices.
It is good to look at music distribution experience - music industry had difficult times because of illegal music distribution, but they were able to reduce their financial losses by creating some good services, where people can easily buy not too expensive content.
It is also possible, considering specifics of 3D-printing, to allow users to print the item only once or for any other limited amount of times.
So, it would prevent manufacturing and further realization of the production by someone else and at the same time, company would be able to let the user to get the product he paid for.
