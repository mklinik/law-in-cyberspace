\section{Background Information}

The act of copying is fundamental to the human nature.  From the days where apes
mimicked each other in how to crack nuts open with a stone, to children
learning from their parents how to use a telephone, Gutenberg's movable types,
Xerox's photocopier, and computers, whose most prominent feature it is to copy
information from the internet to memory, to the hard drive and to the internet
again, we've come a long way.

Over the past decade, a new technology arose which enables mankind to copy not
only information, but physical objects.  In the forseeable future, private
persons will be able to own a machine, roughly as big and expensive as a
kitchen espresso machine, that allows them to fabricate plastic objects like
coat hooks, replacement parts for washing machines, and even firearms, at the
press of a button.

\subsection{The Act of Copying Something}

Copying information: which steps are there?  Scanning documents, transferring,
printing, reproducing etc.  How to copy text/information: Monks copying books
by hand, using the Xerox photocopier, scanning, uploading, downloading and
printing.

How to copy a physical object?  What's similar and what's different?
wood-cutting something by hand.  Scanning, uploading, downloading, printing.
Digital schematics.

If copying is so fundamental and important, why to restrict copying at all?
How does existing law restrict copying of documents/information?  Is it
specific to the thing that is being copied, or is it about the act of copying
"an sich"?  Does existing copyright law apply to physical objects?

3D Printing vs standard industrial casting: There should be laws which forbid
copying products by standard industrial methods.  What's the difference between
3D printing and normal casting?  Do these laws apply?
