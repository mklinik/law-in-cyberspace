\section{Background Information}

The act of copying is fundamental to the human nature.
From the days where apes mimicked each other in how to crack nuts open with a stone, to children learning from their parents how to use a telephone, Gutenberg’s movable types, Xerox’s photocopier, and computers, amongst whose most prominent features it is to copy information from the internet to memory, to the hard drive and to the internet again, we’ve come a long way.

Over the past decades, a new technology arose which enables mankind to copy not only information, but physical objects.
In the foreseeable future, private persons will be able to own a machine, roughly as big and expensive as a kitchen espresso machine, that allows them to fabricate plastic objects like coat hooks, replacement parts for washing machines, and even firearms, at the press of a button.

\subsection{What is 3D Printing?}
A 3D Printer is a machine which, given a blueprint, produces a physical object according to the blueprint.
A blueprint is a computer file that describes the physical extension of an object in enough detail that the object can be reproduced.
There are three roles involved in the legal problem of 3D printing.
There is the ``user'', who uses a 3D printer to create a physical object from a given blueprint. There is the ``designer'', who creates blueprints.
If an existing physical object is involved, the party that may hold some IP right on the object will be called ``manufacturer''.
Take as an example Dave, who owns a car repair shop.
Dave has to repair the car for one of his customers, and he downloads the blueprint for the cap of the windshield wash from the internet.
He then uses his repair shop’s 3D printer to print the cap.
Dave is the user in this example.
The company who built the car may hold IP rights on the cap, so it is our manufacturer.
The person who created the blueprint for the cap, by either using an existing cap as reference, or by examining the tube it must be screwed onto, is the designer.

Designers can create blueprints in various ways, the most common are:
\begin{itemize}
  \item Using a 3D scanner to measure the spatial features of an existing object
  \item Using a CAD (Computer Aided Design) software to create blueprints by hand
\end{itemize}
For the latter option, the designer can use an existing object as reference or create a completely new blueprint without referring to an existing object.

3D printing on its own does not pose a legal problem.
It becomes a legal problem however, when the printed object is protected by 
some IP right, and printing happens against the will of the rights holder.
The act of 3D printing consists of three steps, each of which we will analyze 
independently.
\begin{enumerate}
  \item The designer creates a blueprint.
  \item The blueprint is distributed to the user.
  \item The user prints an object.
\end{enumerate}

If designer and user are the same person, there is no distribution step.
If the designer creates a new blueprint, say some piece of jewellery, without referring to some existing object, there is no legal problem.
We therefore consider the case where the object being printed is a copy of something that already exists and is being manufactured by usual industrial means.

In the following, the three steps are examined in detail, to analyze which IP laws could be violated in each case.
The jurisdiction under consideration is Dutch national law, and EU law when applicable.
We look at possible measures to prevent copying of objects, and how these measures interfere with privacy and data protection laws.

\subsection{Design}
For a blueprint to come into existence, there must be a person creating the blueprint in some way.
We examine now which laws could possibly prohibit the creation of a blueprint from a given physical object.
The object, or certain aspects of it, may be protected by a variety of intellectual property rights.
Of particular interest are the dutch copyright (Auteursrecht), design right (Modellenrecht) and patents (Patentrecht).

Dutch copyright law applies to works of art and science such as texts, talks, software, photographs, films, recorded music, paintings, architecture and journalistic works. %http://wetten.overheid.nl/BWBR0001886/geldigheidsdatum_22-10-2013#HoofdstukI_3_Artikel10
It also applies to sculptures and ``modellen van nijverheid''.
If the object in question is not a work of art, then copyright does not apply. Furthermore, there is to our understanding no clause which forbids detailed description in forms of blueprints.
Copyright law therefore does not preclude the creation of blueprints.

The Benelux design law excludes characteristics from being protected which are determined exclusively by technical function.
%http://wetten.overheid.nl/BWBV0001716/AuthentiekNL/VDRTKS559907/TITELIII/HOOFDSTUK1/Artikel32/geldigheidsdatum_22-10-2013
The most important use case of 3D printing is making spare parts to repair things.
The design of spare parts are usually determined by its function, so design law does not apply.

TODO: patents.


\subsection{Distribution}

As seen in the previous section, one may create a 3D-object by printing it from a blueprint. However, one can obtain a blueprint through various manners, where one of them is P2P file sharing. There have been a few lawsuits on P2P file sharing, but 3D-printing has never been explicitly mentioned, also due to it being a new technology. The question is how the EU legislation should be interpreted with respect to 3D printing. Can a blueprint be seen as intellectual property and how should the regulation with respect to P2P file sharing be interpreted?

Same type of intellectual property?
[case Escher??]

Several foundations such as BREIN and SABAM have filed cases against corporations such as Ziggo and NETLOG. In the opinion of the representatives for music, video and interactive departments, Ziggo (ISP) and NETLOG are used to share files with intellectual property. The prosecutors demanded actions against the illegal P2P file sharing which happened on their services. It is not unthinkable that corporations which use 3D-printing are going to file cases against such service providers.

The question is: “Does a blueprint of a 3D print fall under the same category, that is intellectual property?”. Although the design, a blueprint, of a 3D object is not the same as the object itself, since it is hard to send a 3D object over the digital net, we believe the blueprint does fall under the same category and can be treated the same.

Assuming a music file is treated the same as a blueprint, one might sue corporations which provide services such as P2P file sharing because of this new angle. Several lawsuits have been filed against these kind of service providers. For instance, an action demanded could be to install a monitoring system to filter for P2P file sharing of illegal files. In Case C-360/10 [SABAM vs NETLOG], an argument used was based on directive 2000/31/EC, the preamble, Recital 47: 

“Member States are prevented from imposing a monitoring obligation on service providers (…) of a general nature (…).”

This clearly states that filtering of a general nature is prohibited. In the same directive, Article 15(1), it is stated that: 

“Member States shall not impose a general obligation on providers, (...), to monitor the information which they transmit or store, nor a general obligation actively to seek facts or circumstances indicating illegal activity.”

This means that it is prohibited to seek for illegal activities in their traffic. Although, in Article 3 of Directive 2004/48/EC, which is closely related to directive 2001/29/EC, Article 8 (1), it’s stated that: 

“(1) Member States shall provide for the measures, procedures and remedies necessary to ensure the enforcement of the intellectual property rights covered by this Directive. Those measures, procedures and remedies shall be fair and equitable and shall not be unnecessarily complicated or costly, or entail unreasonable time-limits or unwarranted delays.
(2) Those measures, procedures and remedies shall also be effective, proportionate and dissuasive and shall be applied in such a manner as to avoid the creation of barriers to legitimate trade and to provide for safeguards against their abuse.”

This raises the question what effective, proportionate and dissuasive is, in this case. An injunction to require for service providers to install a filtering system would result in a “serious infringement of the freedom of the service provider to conduct its business since it would require that hosting service provider to install a complicated, costly, permanent computer system at its own expense”, which is not proportionate because there is no fair balance between the protection of the intellectual-property right enjoyed by copyright holders and on the other hand the freedom to conduct business enjoyed by service providers. Moreover, it would also be contrary to what is stated in Article 3(1) of Directive 2004/48/EC, which requires that measures to ensure the respect of intellectual-property rights should not be unnecessarily complicated or costly.

Moreover, installing a filtering system might undermine the freedom of information, since it might also filter lawful content, because of errors, which could lead to the blocking of lawful communications. Furthermore, by introducing a filtering system and therefore distinguishing between lawful and unlawful content might differ between Member States.

As stated in the judgment in Case C-275/06 Promusicae [2008] ECR I-271, the protection of the fundamental right to property, which includes the rights linked to intellectual property, must be balanced against the protection of other fundamental rights, which is in this case is Article 8 of the European Convention on Human Rights, which provides a right to respect for one's "private and family life, his home and his correspondence".

How is the intellectual property and copyright protection in Europe?

As a result, in both the cases SABAM versus NETLOG (Case C-360/10) and SABAM versus Scarlet (Case C-70/10), the result was:

“Directives 2000/31/EC, 2001/29/EC, 2004/48/EC, 95/46/EC and 2002/58/EC, read together and construed in the light of the requirements stemming from the protection of the applicable fundamental rights, must be interpreted as precluding an injunction made against an internet service provider which requires it to install a system for filtering information which is stored on its servers by its service users;
–        all electronic communications passing via its services, in particular those involving the use of peer-to-peer software;
–        which applies indiscriminately to all its customers;
–        as a preventative measure;
–        exclusively at its expense; and
–        for an unlimited period,
which is capable of identifying on that provider’s network the movement of electronic files containing a musical, cinematographic or audiovisual work in respect of which the applicant claims to hold intellectual-property rights, with a view to blocking the transfer of files the sharing of which infringes copyright.”

This means that the installation of a filtering system is prohibited. 

\subsection{Printing}

Private use

Article 2 95/46/EC
(a) 'personal data' shall mean any information relating to an identified or identifiable natural person ('data subject'); an identifiable person is one who can be identified, directly or indirectly, in particular by reference to an identification number or to one or more factors specific to his physical, physiological, mental, economic, cultural or social identity;
Article 6 95/46/EC
1. Member States shall provide that personal data must be:
(a) processed fairly and lawfully;
(b) collected for specified, explicit and legitimate purposes and not further processed in a way incompatible with those purposes. Further processing of data for historical, statistical or scientific purposes shall not be considered as incompatible provided that Member States provide appropriate safeguards;
(c) adequate, relevant and not excessive in relation to the purposes for which they are collected and/or further processed;
(d) accurate and, where necessary, kept up to date; every reasonable step must be taken to ensure that data which are inaccurate or incomplete, having regard to the purposes for which they were collected or for which they are further processed, are erased or rectified;
(e) kept in a form which permits identification of data subjects for no longer than is necessary for the purposes for which the data were collected or for which they are further processed. Member States shall lay down appropriate safeguards for personal data stored for longer periods for historical, statistical or scientific use.
2. It shall be for the controller to ensure that paragraph 1 is complied with.
Article 18 95/46/EC
Obligation to notify the supervisory authority
1. Member States shall provide that the controller or his representative, if any, must notify the supervisory authority referred to in Article 28 before carrying out any wholly or partly automatic processing operation or set of such operations intended to serve a single purpose or several related purposes.
