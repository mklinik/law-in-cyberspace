\section{Advice For Judiciary}
When dealing with cases involving infringement of privacy and the processing of personal data it is of outmost importance that all steps of the process are performed with discretion and care.
When addressed by a claimant of intellectual property rights seeking permission to infringe on the privacy of others the courts are advised to begin by ensuring that any data gathered so far does not fall within the classification of personal data as stated in Article 2.a 95/46/EC.
If the claimant has already gathered personal data then they have infringed on the privacy of people without the permission of the judiciary and in doing so are violating Article 6 95/46/EC.
As such all data gathered should be destroyed and any accusations based on said data should be seen as inadmissible.
The claimant should be made aware of why this is and how they should gather their data in the future so that the new data is impersonal until a court order allows the claimant to do processing of personal data.
It is up to the court to decide if the persons affected by such an infringement should be informed, to allow them to pursue a case of damages caused by the infringement.
However, if the infringement has been light or informing the affected may hinder further proper investigations by the claimant then the court may choose to not inform the affected.

3D printing opens up new challenges for intellectual property law but it still remains the task of the rights holder to present which of their intellectual property rights are being infringed upon by the sharing of blueprints or the printing with the use of said blueprints.
There are many aspects for the claimant to consider and for the courts, in the later stages, to judge.
Some blueprints and their resulting objects may fall under copyright protection, others fall under design protection.
There are also patent rights, trademark rights and even trade secrets that may be part of these blueprints.
It is up to the claimant to show the court if the blueprint alone is enough to prove that these rights have been infringed or if showing that the blueprint is present is a stepping stone to be granted a search warrant to find the printed version if it is only the printed version that infringes their rights.

Once the court agrees that the claimant has shown their intellectual property rights are being infringed upon and that said infringement is proportionate to the infringement of privacy of the persons the claimant seeks permission to perform, e.g. in terms of monetary loss, a court can grant the claimant permission for gathering additional evidence having this legitimate interest in mind.
The courts should however not neglect to inform the claimant that this permission also comes with the obligation to notify the Supervisory Authority to inform them of the extent, intention and duration of their data processing and storing of said data as is their obligation in accordance with Article 18.1 95/46/EC.

After the claimant has carried out their processing, in accordance with the notification they have made to the Supervisory Authority and they have fulfilled their obligations they are free to use the data as evidence for the charges they wish to make.
From there the case leaves the realm of privacy and data protection, as the infringement of privacy and data processing have been done in accordance with the law, and the case becomes an issue of intellectual property law and the claimant can use their evidence to try to prove that their rights have been infringed.
